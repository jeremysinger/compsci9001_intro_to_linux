% !TEX encoding = UTF-8 Unicode
\RequirePackage{pdfmanagement-testphase}
\DocumentMetadata{
testphase = phase-II, % Use Phase-II for intermediate level tagging
pdfversion = 2.0, % Set the PDF version to 2.0 for better compatibility
lang=en-UK % set language for the document
}

\documentclass{article}
\usepackage{xcolor}
\usepackage{hyperref}

\usepackage[scaled]{helvet}
\usepackage{sectsty}
\allsectionsfont{\bfseries\sffamily}

\usepackage{listings}
\lstdefinestyle{BashInputStyle}{
  language=bash,
  morekeywords={mkdir,ls,rm,mv,cp,date,hostname,whoami,zip,unzip,rmdir,pwd,cal},
  basicstyle=\small\sffamily,
%  numbers=left,
%  numberstyle=\tiny,
%  numbersep=3pt,
  frame=tb,
  columns=fullflexible,
  backgroundcolor=\color{yellow!20},
  linewidth=0.9\linewidth,
  xleftmargin=0.1\linewidth,
  literate={-}{-}1,
}

\usepackage{menukeys}

\begin{document}

\noindent
{\Large \textsf{\textbf{Unix Tutorial 1: Files in Linux}}}

\bigskip


Log in to your Unix system with your username and password, then
open a bash shell. On the Glasgow remote Linux setup, you drop into a shell automatically with your ssh login. On a Linux laptop, you should execute command \texttt{gnome-terminal} or \texttt{xterm} to start a bash shell in a window. On a Mac machine, you should execute the \texttt{Terminal} application. On Windows, if you have WSL installed, the \texttt{wsl} command will drop you into a shell.

Type a simple command, e.g.\
\begin{lstlisting}[style=BashInputStyle]
    $ date
\end{lstlisting}
% $
Press Enter after typing in \lstinline{date}. The system should print out the current date and time.
Try a few more simple commands \ldots
\begin{lstlisting}[style=BashInputStyle]
    $ hostname
    $ whoami
    $ cal 2023
\end{lstlisting}
% $

and observe the output.
To quit your bash session, either type \lstinline{exit} or press
\keys{\ctrl + D} together.




Now we are going to practise file manipulations, moving around the directory hierarchy. The special directory \~{} is your home directory, and / is the root directory. Try this:


\begin{lstlisting}[style=BashInputStyle]
    $ cd /
    $ ls
    $ cd ~
    $ ls
\end{lstlisting}

Two more special directories are `dot' \texttt{.} (the current directory)  and `dot dot' \texttt{..} (the parent directory). Another useful command is \texttt{pwd} which shows the current directory you are in.

\textbf{Challenge}: When you first login to your system, how many directories do you have to go up (with \texttt{cd ..}) to reach the root directory? Use \texttt{pwd} to work out when you reach the root.

Now we are going to create some files. We will use the cat command to write a string of characters into plaintext files.

\begin{lstlisting}[style=BashInputStyle]
    $ cat > a.txt
The quick brown fox jumps over the lazy dog.
\end{lstlisting}
% $

You will need to press Enter then \keys{\ctrl + D} after typing the words in the sentence. Here, \keys{\ctrl + D} means `end of file'.

You can run ls again, to show that file a.txt is now in your home directory. Execute some more commands:

\begin{lstlisting}[style=BashInputStyle]
    $ cp a.txt b.txt
    $ ls
    $ mv a.txt fox.txt
    $ cp fox.txt f.txt
\end{lstlisting}

Now we want to create a new directory and move all the files into this directory.

\begin{lstlisting}[style=BashInputStyle]
    $ mkdir tutorial1
    $ ls
    $ mv *.txt tutorial1
    $ cd tutorial1
    $ ls
    $ cd ..
\end{lstlisting}

Now suppose we want to save this directory as a compressed zip archive called files.zip.

\begin{lstlisting}[style=BashInputStyle]
    $ cd ~
    $ zip -r files.zip tutorial1 
    $ ls -lh    
\end{lstlisting}
% $

The zip file has been created in your home directory. It should only be a few bytes in size. Now let's delete the original files and directory that we created.

\begin{lstlisting}[style=BashInputStyle]
    $ cd tutorial1
    $ rm *.txt
    $ cd ..
    $ rmdir tutorial1  
\end{lstlisting}


If you try to execute the rmdir command when the directory is not empty, it will fail and give you an error message. However there is a quicker way to delete a directory and all its contents. Try this:

\begin{lstlisting}[style=BashInputStyle]
    $ unzip files.zip
\end{lstlisting}
% $

to restore the tutorial1 directory and its contents. Use the cd and ls commands to check the files are all present. Now, to delete the tutorial1 directory in one go:

\begin{lstlisting}[style=BashInputStyle]
    $ cd ~
    $ rm -rf tutorial1
\end{lstlisting}

The r flag stands for `recursive' and f for `force'. Use this command with caution! In his book \textit{Creativity Inc.}, Ed Catmull describes how most of the graphics from \textit{Toy Story 2} were wiped from Pixar's filestore by careless use of the rm command. Google for \texttt{rm toy story 2} to find out the details.

\section*{Questions}

\begin{enumerate}
\item If \texttt{cd ..} moves to the parent directory, then what is the parent directory of root \texttt{/} ?
\item Why is the following command a \emph{very bad idea?} \texttt{rm -rf /}
\item In your bash prompt, what does the \keys{\arrowkeyup}  cursor arrow do? What about \keys{\arrowkeydown}?
\item Some commands (like \texttt{ls} and \texttt{rm}) take extra option flags or switches. See if you can find out the switches for the following (using Google or \texttt{man}):
  \begin{itemize}
  \item listing all files in a directory in order of modification, with most recently modified first?
  \item listing all files in a directory in order of modification, with most recently modified last?
  \item removing a file, with an interactive check where the user has to type \texttt{y} if they really want to remove the file?
  \end{itemize}
\end{enumerate}


\section*{Further Reading}

For further investigation today, use the \texttt{man} command to find out 
the flags for all the commands you have already executed.
For instance, you might run \texttt{man cat} to Find out how the cat command works in more detail---you can use cat both to print out existing text files, to create new ones and to append to existing files.
You press \keys{q} to quit the man page viewer and return to the terminal.

There are plenty of helpful file handling tutorials online, some examples
below:
\begin{itemize}
\item \url{https://ryanstutorials.net/linuxtutorial/filemanipulation.php}
\item \url{https://www.redhat.com/sysadmin/navigating-filesystem-linux-terminal}
\item \url{https://www.redhat.com/sysadmin/Linux-file-navigation-commands}
\end{itemize}
  
  

\end{document}
