\documentclass{article}
\usepackage{xcolor}
\usepackage{hyperref}

\usepackage[scaled]{helvet}
\usepackage{sectsty}
\allsectionsfont{\bfseries\sffamily}

\usepackage{listings}
\lstdefinestyle{BashInputStyle}{
  language=bash,
  morekeywords={mkdir,ls,rm,mv,cp,date,hostname,whoami,zip,unzip,rmdir},
  basicstyle=\small\sffamily,
%  numbers=left,
%  numberstyle=\tiny,
%  numbersep=3pt,
  frame=tb,
  columns=fullflexible,
  backgroundcolor=\color{yellow!20},
  linewidth=0.9\linewidth,
  xleftmargin=0.1\linewidth,
  literate={-}{-}1,
}

\usepackage{menukeys}

\begin{document}

\noindent
{\Large \textsf{\textbf{Unix Tutorial X: Glasgow Linux Quirks}}}

\bigskip

This worksheet will give you some useful hints and tips
for managing your account on the Glasgow Linux servers
in the Boyd Orr labs.

These machines are running CentOS Linux 7.3.
This is a Red Hat variant of Linux. You can verify this by
executing the command \lstinline{cat /etc/redhat-release}
and observing the output.


\subsection*{File Space}

Your home directory should appear the same, whichever lab machine
you log into. This is because your home directory is a network
mapped drive. You have a file space limit on this directory. You can
find out what this limit is, and how much space you are using, with the
\lstinline{quota} command. It should show something like:

{\small
\begin{verbatim}
Disk quotas for user jsinger (uid 3252): 
     Filesystem  blocks   quota   limit   grace   files   quota   limit   grace
unix5:/export/users/grad
                9811584  20000000 22000000          285508       0       0        
\end{verbatim}
}

Given 1K block size, this means I have a quota of 20GB, which is probably more than you have :-(

To find out which files and directories are taking up lots of space,
run the command \lstinline{du -hsc * | sort -h} and the final items should be the largest.
Note that this does not include `hidden' directories, which start with a . character. To see these instead, you need to modify the command to \lstinline{du -hsc .[!.]* | sort -h} .


\subsection*{Installing Programs}

If you are administering your own Linux machine, you can install
programs from package repositories using dnf, apt-get or similar utilities. However this is not possible on the Boyd Orr machines, since
you do not have root access. Instead, you should download source code
tarballs and build programs from source. Suppose you download foo.tar.gz and want to install it in your system. You would go through a
sequence of commands like this:

\begin{lstlisting}[style=BashInputStyle]
  tar xvzf foo.tar.gz
  cd foo
  ./configure --prefix=/tmp/
  make
  make install
\end{lstlisting}

where the prefix flag specifies the directory where you want to instal the utility. Note that configure expects bin, include, lib and man directories inside the prefix directory. Ask a tutor for more details.
You will want to have the prefix bin directory in your PATH so you can execute the program, when it is installed.

Try this out with a simple utility, by following instructions at
\url{http://www.ee.surrey.ac.uk/Teaching/Unix/unix7.html}.

\subsection*{Remote Access}

You can login to the Boyd Orr machines using ssh, via an ssh session
on our gateway server, sibu.dcs.gla.ac.uk.
Use your standard login credentials to access sibu.

If you want to forward web traffic from a dcs machine to an external
machine, you can use ssh forwarding from sibu. So, suppose I want to access port 80 (default web server port) or port 443 (default https port) on machine 130.209.240.41 --- I can do this via sibu with the following command:

\begin{lstlisting}[style=BashInputStyle]
# http variant ...
ssh -L 8001:130.209.240.41:80 -N jsinger@sibu.dcs.gla.ac.uk
# https variant ...
ssh -L 8001:130.209.240.41:443 -N jsinger@sibu.dcs.gla.ac.uk
\end{lstlisting}

This allows me to access the remote machine by pointing my web browser at \url{http://localhost:8001} or \url{https://localhost:8001}. Try it for the https variant on this IP address above--- where do you get to?
Ssh tunnelling like this is very useful
when you want to access SoCS servers from outside the university network. Make sure you know how to do this for your PSD course!

\subsection*{Quitting Programs}

\begin{itemize}
\item To quit from an interactive shell, type \texttt{exit} or press
\keys{\ctrl + D} together.
\item To quit from a running shell utility, type \keys{\ctrl + C} together.
\item To quit a GUI program, run the command \texttt{xkill} from a console, then click the skull-and-crossbones mouse cursor over the GUI window of the program you want to quit.
\item To quit a running process, you can find its process id in a terminal with the \texttt{ps} command. Then run \lstinline{kill -9 xxx} where xxx is the integer process id number.
\item To quit a scrolling text viewer, perhaps more, less or man, type \keys{q}.
\item To restart the system when you have problems, type \keys{CTRL+ALT+F2} together to get a non-graphical login prompt. Then type
  \keys{CTRL+ALT+DEL} to initiate a restart.
\end{itemize}

\section*{Further Reading}

For further investigation today, use the man command to find out 
the flags for all the commands you have already executed.

Also talk to the tutors to find out about other useful commands.

\end{document}
