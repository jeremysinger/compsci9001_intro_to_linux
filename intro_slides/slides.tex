\documentclass{beamer}

\usepackage{url}
\begin{document}

\title{Linux Fundamentals}
\author{Jeremy Singer}

\frame{\titlepage}

\begin{frame}
\frametitle{Course objectives}
\begin{enumerate}
\item gain familiarity with Unix systems
\item find out about useful tools and techniques
  \item find out how to find out about useful \ldots
\end{enumerate}
\end{frame}

\begin{frame}
\frametitle{Learning sessions}
\begin{enumerate}
\item one-off lecture (now)
\item videos on youtube (via moodle)
\item self-study labs (x3, 12--1, Tue--Thu)
\item labs with tutors (x4, 2--3, Mon-Thu)
\end{enumerate}
\end{frame}

% three surprising facts about Linux

\begin{frame}
\frametitle{Linux is everywhere}

From supercomputers to smartphones, from internet routers to Raspberry Pi
\end{frame}


\begin{frame}
\frametitle{Linux is developed everywhere}



\end{frame}

\begin{frame}
\frametitle{Linux started as a one-person student project}
  {\tiny
    From: \url{torvalds@klaava.Helsinki.FI} (Linus Benedict Torvalds)
    Newsgroups: comp.os.minix
    Subject: What would you like to see most in minix?
    Summary: small poll for my new operating system
    Date: 25 Aug 91 20:57:08 GMT
    Organization: University of Helsinki

    Hello everybody out there using minix –

    I’m doing a (free) operating system (just a hobby, won’t be big and
    professional like gnu) for 386(486) AT clones. This has been brewing
    since april, and is starting to get ready. I’d like any feedback on
    things people like/dislike in minix, as my OS resembles it somewhat
    (same physical layout of the file-system (due to practical reasons)
    among other things).

    I’ve currently ported bash(1.08) and gcc(1.40), and things seem to work.
    This implies that I’ll get something practical within a few months, and
    I’d like to know what features most people would want. Any suggestions
    are welcome, but I won’t promise I’ll implement them :-)

    Linus 

    PS. Yes – it’s free of any minix code, and it has a multi-threaded fs.
    It is NOT protable (uses 386 task switching etc), and it probably never
    will support anything other than AT-harddisks, as that’s all I have :-(.

}
\end{frame}

% useful information

\begin{frame}
  \frametitle{Useful info}
  \begin{itemize}
  \item login with your student id and last 8 digits of library barcode
  \item yell at me if you can't login
    \item fire up a terminal session --- \texttt{gnome-terminal} or similar
  \item access SoCS Linux remotely via server sibu.dcs.gla.ac.uk
  \end{itemize}
\end{frame}


\end{document}
